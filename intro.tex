\chapter{Introduction}
First we shall introduce the most important characters in our following exploration. The ideas and definitions here will be recurring regularly as we examine them from different perspectives and using different tools.

\begin{definition}[Dynamical System (DS)]
	A triple $(P,E, \mathcal{F})$, with
	\begin{itemize}
		\item $P :$ the phase space for the dynamical variable $x\in P$,
		\item  $E:$ base space of the evolutionary variable (e.g. time) $t \in E$,
		\item $\mathcal{F}: $ the evolution rule (deterministic) which defines the transition from one state to the next.
\end{itemize}
\end{definition}
The two main types of evolutionary variable spaces are
\begin{enumerate}
	\item Discrete dynamical systems (DDS) $t\in E=\mathbb{Z}$ with trajectory $\{x_0, x_1, \ldots\}$,
	\item Continuous dynamical systems (CDS) $t\in E=\mathbb{R}$ with trajectory $\{x_t\}_{t \in \mathbb{R}}$.
\end{enumerate}
Corresponding to these there are various types of evolution rules
\begin{enumerate}
	\item In a DDS we have iterated mappings 
	\begin{align}
		\boxed{x_{n+1} = F(x_n , n).}	
	\end{align}
	If there is no explicit dependence on $n$, i.e. $\frac{\partial F}{\partial n} = 0$, then 
	\begin{align}
		\boxed{ x_{n+1} F(x_n) = F(F(x_{n-1})) = \underbrace{F \circ \ldots \circ F}_{n+1 \textrm{ times} }(x_0) = F^{n+1}(x_0).}
	\end{align}
\begin{ex}[]
	\begin{figure}[h]
	\centering
	\includegraphics[width = \textwidth]{figures/intro/1DDS.png}
\end{figure}
\end{ex}

\item In a CDS we have a first order system of ordinary differential equations (ODE)
	\begin{align}
		\boxed{
			\dot{x} = f(x,t)
		}
	\end{align}
	for $x\in P$ and $t \in E$. This yields the initial value problem (IVP):
	\begin{align}
		\begin{dcases}
			\dot{x} = f(x,t) \\
			x(t_0) = x_0
		\end{dcases}
	\end{align}
	\begin{figure}[h]
	\centering
	\includegraphics[width = 0.8\textwidth]{figures/intro/2CDS.png}
	\end{figure}
	
	Assuming there exists a unique solution $\varphi(t; t_0, x_0)$ with $\dot{\varphi} = f(\phi,t)$ and $\varphi(t_0)= x_0$, then the following flow map is well defined
	\begin{align}
		\boxed{
		F_{t_0}^{t}(x_0) := \varphi(t; t_0, x_0).}
	\end{align}
	Such an $F_{t_0}^{t}$ has nice properties
	\begin{itemize}
		\item $F_{t_0}^{t}$ is as smooth as $f(x,t)$,
		\item $F_{t_0}^{t_0} = I$ and $F_{t_0}^{t_2} = F_{t_1}^{t_2} \circ F_{t_0}^{t_1}$ (group property),
		\item $\left(F_{t_0}^{t}\right)^{-1} = F_{t}^{t_0}$ exists and is smooth.
\end{itemize}
A special case of this is the autonomous system 
\begin{align}
	\boxed{\dot{x} = f(x).}	
\end{align}
The autonomy of a system implies
\begin{align}
	x(s,t_0, x_0) = x(\underbrace{s-t_0}_{t}, 0, x_0) \stackrel{!}{=} x(t,x_0).
\end{align}
And the induced flow map in this case is the one-parameter family of maps
\begin{align}
	\boxed{ F^{t} = F_{0}^{t}: x_0 \mapsto x(t,x_0).}
\end{align}
\end{enumerate}
\begin{ex}[Logisitic Equation]
	For a resource-limited population, we have the following dynamic system for $a> 0$, $b> 0$, and the population $x\in \mathbb{R}_+ \cup \{0\}$
	\begin{align}
		\dot{x} = ax(b-x).
	\end{align}
	In this case we have $E=\mathbb{R}$ and $\mathcal{F} = \{F^{t}\}_{t=-\infty }^{+\infty }$. This system has globally existing unique solutions (see later).	
	\begin{figure}[h]
		\centering
		\includegraphics[width=0.4\textwidth]{figures/intro/3RHS.png}	
		\hspace{0.05\textwidth}
		\includegraphics[width=0.4\textwidth]{figures/intro/4solutions.png}
		\caption{Left: Analysis of the right hand side. Right: Evolution in the extended phase space $P \times \mathbb{R}$.}
	\end{figure}

\end{ex}

\begin{ex}[Pendulum]
Given the equation of motion
\begin{align}
	ml^2 \ddot{\varphi} = -mgl \sin(\varphi).
\end{align}
We let $x_1 = \varphi$ and $x_2 =\dot{\varphi}$ to transform into the first-order ODE form
\begin{align}
	\begin{dcases}
		\dot{x_1} = x_2 \\
		\dot{x_2} = - \frac{g}{l} \sin (x_1).
	\end{dcases}
\end{align}
Thus we have 
\begin{align}
x = 
\begin{pmatrix}
	x_1 \\ x_2
\end{pmatrix}; \quad
f(x) = 
\begin{pmatrix}
	x_2 \\ - \frac{g}{l}\sin(x_1)	
\end{pmatrix}.
\end{align}
Qualitatively analysis gives the following facts
\begin{itemize}
	\item $x_1, x_2) = (0,0)$ and $(x_1, x_2) = (\pi , 0)$ are zeros of $f$,
	\item Energy is conserved, hence both small and large amplitude oscillations are expected,
	\item We have the symmetries $(x_1, x_2, t) \mapsto (x_1, -x_2, -t)$ and $(x_1, x_2, t) \mapsto (-x_1, x_2, -t)$.
\end{itemize}
\begin{figure}[h]
	\centering
	\includegraphics[width=0.4\textwidth]{figures/intro/6pendulum_symmetries.png}
	\hspace{0.05\textwidth}
	\includegraphics[width=0.4\textwidth]{figures/intro/5pendulum.png}
	\caption{Left: The symmetries of the dynamic system. Right: Phase portrait of the pendulum. The blue trajectories are separatrix.}
\end{figure}
\begin{definition}
	A separatrix connects fixed points, is unobservable by itself, and separates regions of similar behavior.
\end{definition}

\end{ex}

\begin{ex}[Exploit geometry of phase space for analysis]
	Two bikes \emph{can} make it from $A$ to $B$ on different routes without exceeding distance $D$. Assume two trucks are trying to make it between $A$ and $B$, on different roads in the opposite direction, carrying load of width $D$. Can the trucks make it without hitting each other?	
	\begin{figure}[h]
		\centering
		\includegraphics[width=0.5\textwidth]{figures/intro/7routes.png}
		\hspace{0.05\textwidth}
		\includegraphics[width=0.3\textwidth]{figures/intro/8truck_geometry.png}
		\caption{Left: An example of the two bike routes. Right: Blue represents the phase trajectory of the two biker, red represents the phase trajectory of the two trucks.}
	\end{figure}

The two trajectories must intersect by continuity, thus at that point the trucks must be at the same positions as the bikes, implying they are within distance $D$. Therefore the trucks must crash!	
\end{ex}


