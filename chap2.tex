\chapter{Stability of fixed points}
\section{Basic definitions}
Consider
\begin{align}
	\dot{x}=f(x,t),\ x \in \mathbb{R}^{n},\ f\in C^{1}.
\end{align}
Assume that $x=0$ is a fixed point, i.e. $f(0,t) = 0$ for all $t \in \mathbb{R}$. If the fixed point is originally at $x+0\neq 0$, shift it to zero by letting $\tilde{x}:=x-x_0$, therefore 
\begin{align}
	\dot{\tilde{x}} = \dot{x} = f(x_0 + \tilde{x}, t) = \tilde{f}(\tilde{x}, t).
\end{align}
\textbf{Question} How does the dynamical system behave near its equilibrium state?

\begin{definition}[Lyapupnov Stability]
	$x=0$ is stable if for all $t_0$, for all $\epsilon>0$ small enough, there exists a $\delta=\delta(t_0, \epsilon)$, such that for all $x_0 \in \mathbb{R}^{n}$ with $\|x_0\| \leq \delta$, we have 
	\begin{align}
		\left \| x(t;t_0, x_0) \right\| \leq \epsilon \quad \forall t \geq t_0.
	\end{align}
\begin{figure}[h]
	\centering
	\includegraphics[width=0.7\textwidth]{figures/ch2/1lyapunov_stability.png}
	\caption{An example such a $\delta$, $B(r)$ represents the $n$-dimensional ball of radius $r$.}
\end{figure}
\end{definition}

\begin{ex}[Stability of lower  equilibrium of the pendulum]
	Recall we have $\ddot{\varphi} + \sin(\phi) = 0$, and we transform this into a first order ODe by setting $x_1 = \varphi$ and $x_2 = \dot{\varphi}$ to obtain
	\begin{align}
		\begin{dcases}
			\dot{x_1} = x_2 \\
			\dot{x_2} = -\sin(x_1).
		\end{dcases}
	\end{align}
	For small $\epsilon>0$, this geometric procedure gives a $\delta(\epsilon)>0$ such that the definition of stability is satisfied for $x=0$. Therefore $x=0$ is (Lyapunov) stable.
	\begin{figure}[h!]
		\centering
		\includegraphics[width=0.5\textwidth]{figures/ch2/2pendulum_stability.png}
		\caption{Stability of lower equilibrium for the pendulum, here $0<\epsilon<\pi $.}
	\end{figure}
\end{ex}

\begin{definition}[Asymptotic stability]
$x=0$ is asymptotically stable if
\begin{enumerate}
	\item it is stable,
	\item for all $t_0$, there exists $\delta_0(t_0)$ such that for every $x_0$ with $\| x_0 \| \leq \delta_0$ we have
		\begin{align}
			\boxed{\lim_{t\to \infty } x(t; t_0, x_0) = 0.}
		\end{align}
\end{enumerate}
\begin{figure}[h]
	\centering
	\includegraphics[width=0.7\textwidth]{figures/ch2/3asymp_stability.png}
	\caption{An example for an asymptotically stable fixed point (yellow trajectory).}
\end{figure}
\end{definition}

\begin{definition}[Domain of attraction]
	The set of all $x_0$'s for which
	\begin{align}
		\boxed{\lim_{t\to \infty }x(t;t_0, x_0)=0. }	
	\end{align}
	
\end{definition}

\begin{ex}[Damped pendulum]
	We have the equation of motion
	\begin{align}
		\ddot{\varphi} + c \dot{\varphi} + \sin(\varphi) = 0,\quad c>0.
	\end{align}
	Transforming into a first-order ODE with $x_1 = \varphi$ and $x_2 = \dot{\varphi}$ gives
	 \begin{align}
		\begin{dcases}
			\dot{x_1} = x_2\\ \dot{x_2} = -cx_2 - \sin(x_1).
		\end{dcases}
	\end{align}
The total energy is given by
\begin{align}
	E = \frac{1}{2}x_2^2 + \left( 1 - \cos(x_1) \right). 
\end{align}
Further we have the rate of energy change
\begin{align}
	\frac{d}{dt} E(x_1(t), x_2(t)) = x_2 \left(\dot{x_2} + \sin(x_1) \right) = -c x_2^{2}.
\end{align}
\begin{figure}[h]
	\centering
	\includegraphics[width=0.35\textwidth]{figures/ch2/4damped_pendulum.png}
\end{figure}

Therefore, along trajectories energy decreases monotonically. By the $C^0$ dependence on initial conditions, the trajectories remain close to the undamped oscillations for small $c>0$. We conclude that trajectories are inward spirals for $c>0$ small. $x=0$ is still Lyapunov stable, but asymptotic stability does not yet follow (is the limit of $x(t)$ equal to 0?).
\begin{remark}[Lasalle's invariance principle]
	This conclusion follows rigorously from Lasalle's invariance principle, namely if we assume that $\dot{x}=f(x)$, $f \in C^1$, and that there exists a $V\in C^1$ with 
	\begin{align}
		\dot{V} = \frac{dV(x(t))}{dt} \leq 0.	
	\end{align}
	Then the set of accumulation points for any trajectory is contained in the set of trajectories that stay within the set $I=\{x \in \mathbb{R}^{n}:\ \dot{V}(x) = 0\}$.
\end{remark}
\end{ex}

\begin{ex}[]
	We are given the following dynamic system in polar coordinates
	\begin{align}
		\begin{dcases}
			\dot{r} = r(1-r) \\ \dot{\theta} = \sin^2\left( \frac{\theta}{2} \right).
		\end{dcases}
	\end{align}
	Note that $r=0$ is a fixed point, the set $r=1$ is an invariant circle, and the set $\theta=0$ is an invariant set. An invariant set is a set in which if the dynamic system is 'started' on the set, it remains in the set for all time. Examining the radial evolution reveals that the that the equation of motion decouples. We see that $\dot{\theta}\geq 0$, so rotation is either positive or null.
	\begin{figure}[h]
		\centering
		\includegraphics[width=0.5\textwidth]{figures/ch2/5polar_cds.png}
		\caption{Phase portrait of the dynamic system, with arrows pointing to the two unstable equilibria.}
	\end{figure}

	However, we have that $p=(1,0)$ is an example of an attractor: a set with an open neighborhood of points that all approach the set as $t\to \infty $.	
\end{ex}

\begin{definition}[Invariant set]
	$S \subset P$ is an invariant set for the flow map $F^{t}:P \to P$ if $F^{t}(S) =S$ for all $t \in \mathbb{R}$.
\end{definition}

\begin{definition}[Unstable point]
	A fixed point $x=0$ is unstable if it is not stable.
\end{definition}

\begin{remark}[]
	We can negate a mathematical statement by using the reverse relational operators outside the statements involving these operators i.e. $ \exists \to \forall $ and $\forall \to \exists $. For example we have for continuity $\forall \epsilon\ \exists \delta:\ |f(x) - f(y)| < \epsilon$ if $|x-y|<\delta$, meanwhile for discontinuity we have  $\exists \epsilon:\ \forall \delta:\ |f(x) - f(y)| \not < \epsilon$ for $|x-y|< \delta$.

	In our case for stability we have
	\begin{align}
		\forall \epsilon,t_0: \quad \exists \delta>0: \quad \forall x_0  \textrm{ with } |x_0| < \delta: \quad |x(t)|\leq \epsilon \quad \forall t\geq t_0.
	\end{align}
Meanwhile for unstability
\begin{align}
	\exists \epsilon,t_0:\quad \underbrace{\forall \delta>0}_{ \textrm{"for arbitrarily small"} }:\quad \exists x_0  \textrm{ with } |x_0|<\delta: \quad |x(t)|>\epsilon \quad \underbrace{\exists t\geq t_0}_{ \textrm{"for some"} }.
\end{align}
\begin{figure}[h]
	\centering
	\includegraphics[width=0.5\textwidth]{figures/ch2/6unstable_def.png}
\end{figure}
\end{remark}

\begin{remark}[]
	By $C^0$ dependence on initial conditions, if $x(t;t_0,x_0)$ leaves $B(\epsilon)$, then for $\tilde{x}_0$ close enough to $x_0$, $x(t;t_0,\tilde{x}_0)$ also leaves $B(\epsilon)$. Therefore if the measure of such trajectories in nonzero, the instability is observable!
\end{remark}

\begin{ex}[Unstable fixed point of pendulum]
	We have that infinitely many trajectories converge to the fixed point, yet it is still unstable. In fact, the converging trajectories form a measure-zero set, thus the stability near the unstable equilibrium is unobservable.
	\begin{figure}[h]
		\centering
		\includegraphics[width=0.6\textwidth]{figures/ch2/7unstable_pendulum.png}
		\caption{The phase portrait around the unstable fixed point of the pendulum, with the stable trajectories (red).}
	\end{figure}
	
\end{ex}
\newpage
\section{Stability based on linearization}
In the following section we shall always assume that our system is autonomous. We will have the following setup
\begin{align*}
	\dot{x}=f(x),\quad f\in C^1,\quad x=
	\begin{pmatrix}
		x_1\\ \vdots \\ x_n
	\end{pmatrix}\in \mathbb{R}^{n}, \quad 
p = 
\begin{pmatrix}
	p_1 \\ \vdots \\ p_n 
\end{pmatrix}
\in \mathbb{R}^{n}. \numberthis \label{eq:star}
\end{align*}

