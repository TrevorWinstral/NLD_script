%% LyX 2.3.6.2 created this file.  For more info, see http://www.lyx.org/.
%% Do not edit unless you really know what you are doing.
\documentclass[english]{article}
\usepackage[T1]{fontenc}
\usepackage[latin9]{inputenc}
\pagestyle{plain}
\setcounter{secnumdepth}{0}
\usepackage{float}
\usepackage{amsmath}
\usepackage{amssymb}
\usepackage{graphicx}

\makeatletter
%%%%%%%%%%%%%%%%%%%%%%%%%%%%%% User specified LaTeX commands.


%%%%%%%%%%%%%%%%%%%%%%%%%%%%%%%%%%%%%%%%%%%%%%%%%%%%%%%%%%%%%%%%%%%%%%%%%%%%%%%%%%%%%%%%%%%%%%%%%%%%%%%%%%%%%%%%%%%%%%%%%%%%%%%%%%%%%%%%%%%%%%%%%%%%%%%%%%%%%%%%%%%%%%%%%%%%%%%%%%%%%%%%%%%%%%%%%%%%%%%%%%%%%%%%%%%%%%%%%%%%%%%%%%%%%%%%%%%%%%%%%%%%%%%%%%%%
\usepackage{amsfonts}

\setcounter{MaxMatrixCols}{10}
%TCIDATA{TCIstyle=LaTeX article (bright).cst}

%TCIDATA{OutputFilter=LATEX.DLL}
%TCIDATA{Version=5.00.0.2606}
%TCIDATA{<META NAME="SaveForMode" CONTENT="1">}
%TCIDATA{BibliographyScheme=Manual}
%TCIDATA{Created=Thursday, September 13, 2001 12:48:34}
%TCIDATA{LastRevised=Friday, April 28, 2006 16:52:00}
%TCIDATA{<META NAME="GraphicsSave" CONTENT="32">}
%TCIDATA{<META NAME="DocumentShell" CONTENT="Exams and Syllabi\SW\Assignment">}
%TCIDATA{Language=American English}

\setlength{\topmargin}{-1.0in}
\setlength{\textheight}{9.25in}
\setlength{\oddsidemargin}{0.0in}
\setlength{\evensidemargin}{0.0in}
\setlength{\textwidth}{6.5in}
\def\labelenumi{\arabic{enumi}.}
\def\theenumi{\arabic{enumi}}
\def\labelenumii{(\alph{enumii})}
\def\theenumii{\alph{enumii}}
\def\p@enumii{\theenumi.}
\def\labelenumiii{\arabic{enumiii}.}
\def\theenumiii{\arabic{enumiii}}
\def\p@enumiii{(\theenumi)(\theenumii)}
\def\labelenumiv{\arabic{enumiv}.}
\def\theenumiv{\arabic{enumiv}}
\def\p@enumiv{\p@enumiii.\theenumiii}


\parindent=0pt

\makeatother

\usepackage{babel}
\begin{document}
\begin{center}
\textbf{151-0530-00L, Spring, 2022}
\par\end{center}

\begin{center}
\textbf{Nonlinear Dynamics and Chaos II}
\par\end{center}

\begin{center}
\textbf{Homework Assignment 4}
\par\end{center}

\begin{center}
Due: Friday, May 20\\
Please email PDF file to Dr. Mattia Cenedese <mattiac@ethz.ch>
\par\end{center}

\bigskip{}

\begin{enumerate}
\item Compute the Lyapunov-type numbers $\nu(p)$ and $\sigma(p)$ in the
example
\begin{align*}
\dot{x} & =-x(1-x^{2}),\\
\dot{y} & =-by,
\end{align*}
for all points $p\in M_{0}$, with the parameter $b\in\mathbb{R}^{+}$
and with overflowing-invariant manifold $M_{0}$ defined as
\[
M_{0}=\left\{ (x,y)\in\mathbb{R}^{2}:\,\,\,y=0,\,\,x\in\left[-3/2,3/2\right]\right\} .
\]
 (\emph{Hint}: Use the operators $A_{t}(p)$ and $B_{t}(p)$ defined
in class). 
\item The stable and unstable manifolds of a normally hyperbolic invariant
manifold $M$ turn out to admit a delicate internal structure, an
\emph{invariant foliation, }which is useful in determining the exact
asymptotic behavior of trajectories in $W^{u}(M)$ and $W^{s}(M)$.

\qquad{}More specifically, if $M\subset\mathbb{R}^{n}$ is a compact,
$C^{r}$ smooth, $k$-dimensional, $r$-normally hyperbolic invariant
manifold with boundary, and $\dim\left[W^{s}(M)\right]=k+s$, then
$W^{s}(M)$ has the following properties (some of which are sketched
in Fig. 1.):
\begin{figure}[H]
\begin{centering}
\includegraphics[width=0.4\textwidth,bb = 0 0 200 100, draft, type=eps]{../Homeworks2017/fiber.eps}\caption{The geometry of stable fibers}
\par\end{centering}
\end{figure}

(i) Near $M$, the stable manifold $W^{s}(M)$ can be written as
\[
W_{\mathrm{loc}}^{s}(M)=\cup_{p\in M}\,\,f^{s}(p),
\]
where $f^{s}(p)$ is a $C^{r}$ smooth, $s$-dimensional submanifold
of $W_{\mathrm{loc}}^{s}(M)$ for which $f^{s}(p)\cap M=p$. We refer
to the point $p$ on $M$ as the base point of the \emph{stable fiber
}$f^{s}(p)$.

(ii) The stable fiber $f^{s}(p)$ is tangent to $N_{p}^{s}M$, the
local section of the stable subbundle $N^{s}M.$

(iii) The stable fibers form a positively invariant family, i.e.,
$F^{t}(f^{s}(p))\subset f^{s}(F^{t}(p))$. In words, stable fibers
are mapped into stable fibers by the flow map, although individual
stable fibers are not invariant under the flow.

(iv) There exist positive constants $C_{s}$ and $\lambda_{s}$, such
that for any $q\in f^{s}(p)$, we have $\left\vert F^{t}(q)-F^{t}(p)\right\vert <C_{s}e^{-\lambda_{s}t}.$
In other words, trajectories intersecting a stable fiber will exponentially
converge to the trajectory on $M$ that passes through the base point
of that stable fiber.

(v) For any $q\in f^{s}(p)$ and $\hat{q}\in f^{s}(\hat{p}),$ we
have
\[
\frac{\left\Vert F^{t}(q)-F^{t}(p)\right\Vert }{\left\Vert F^{t}(\hat{q})-F^{t}(p)\right\Vert }\rightarrow0,
\]
as $t\rightarrow\infty$, unless $p=\hat{p}.$ In other words, out
of all the trajectories that may converge to the positive half-trajectory
\[
\gamma(p)=\left\{ F^{t}(p)\right\} _{t\geq0},
\]
the trajectories starting from the stable fiber $f^{s}(p)$ converges
at the fastest rate. One therefore obtains a local stable manifold
\[
W_{\mathrm{loc}}^{ss}(\gamma(p))=\cup_{\tilde{p}\in\gamma(p)}\,f^{s}(\tilde{p})
\]
for any trajectory $\gamma(p)$ on the manifold $M$. (The full stable
manifold $W_{\mathrm{loc}}^{s}(\gamma(p))$ may be larger than $W_{\mathrm{loc}}^{ss}(\gamma(p))$,
because $\gamma(p)$ may also attract trajectories within $M$.)

(vi) $f^{s}(p)\cap f^{s}(\hat{p})=\emptyset,$ unless $p=\hat{p}.$
In other words, stable fibers with different base points do not intersect.

(vii) A stable fiber $f^{s}(p)$ is a $C^{r-1}$ smooth function of
its base point $p$.

(viii) Stable fibers $C^{r}$-smoothly persist under small $C^{1}$
perturbations of the dynamical system.

The local unstable manifold $W_{\mathrm{loc}}^{u}(M)$ has a similar
invariant foliation 
\[
W_{\mathrm{loc}}^{u}(M)=\cup_{p\in M}\,\,f^{u}(p),
\]
with appropriate properties in backward time. (For more information,
see S.\ Wiggins, \emph{Normally Hyperbolic Invariant Manifolds in
Dynamical\ Systems}, Springer 1994)

Consider now the three-dimensional nonlinear dynamical system
\begin{eqnarray*}
\dot{x} & = & -\varepsilon\left(x+y^{2}\right),\\
\dot{y} & = & -y,\\
\dot{z} & = & z,
\end{eqnarray*}
with the small parameter $\varepsilon\geq0$.

(a) Show that the set $M_{0}=\left\{ y=z=0,\,x\in\left[-1,1\right]\right\} $
is a normally hyperbolic invariant manifold for $\varepsilon=0.$

(b) Find the manifold $M_{\varepsilon}$ into which $M_{0}$ perturbs
for small $\varepsilon>0$.

(c) Using the property (v), show that for any base point $p\in$ $W_{\mathrm{loc}}^{s}(M)$,
the corresponding stable fiber is the nonlinear surface.
\[
f^{s}(p)=\left\{ \left(x,y,z\right)~|~x=x_{p}+\frac{\varepsilon}{2-\varepsilon}y^{2},\,z=0\right\} .
\]

(d) Find a similar expression for the unstable fibers $f^{u}(p)$.

(e) Verify explicitly the properties of the stable fibers listed in
(i)-(vii) in this example.

(e) For any trajectory $\gamma$ in $M_{\varepsilon}$, find explicit
expressions for $W_{\mathrm{loc}}^{ss}(\gamma)$, $W_{\mathrm{loc}}^{uu}(\gamma),$
$W_{\mathrm{loc}}^{s}(\gamma)$, and $W_{\mathrm{loc}}^{u}(\gamma).$ 
\end{enumerate}

\end{document}
