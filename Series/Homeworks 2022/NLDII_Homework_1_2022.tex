%% LyX 2.3.6.2 created this file.  For more info, see http://www.lyx.org/.
%% Do not edit unless you really know what you are doing.
\documentclass[english]{article}
\usepackage[T1]{fontenc}
\usepackage[latin9]{inputenc}
\pagestyle{plain}
\setcounter{secnumdepth}{0}
\usepackage{amsmath}
\usepackage{amssymb}

\makeatletter
%%%%%%%%%%%%%%%%%%%%%%%%%%%%%% User specified LaTeX commands.


%%%%%%%%%%%%%%%%%%%%%%%%%%%%%%%%%%%%%%%%%%%%%%%%%%%%%%%%%%%%%%%%%%%%%%%%%%%%%%%%%%%%%%%%%%%%%%%%%%%%%%%%%%%%%%%%%%%%%%%%%%%%%%%%%%%%%%%%%%%%%%%%%%%%%%%%%%%%%%%%%%%%%%%%%%%%%%%%%%%%%%%%%%%%%%%%%%%%%%%%%%%%%%%%%%%%%%%%%%%%%%%%%%%%%%%%%%%%%%%%%%%%%%%%%%%%
\usepackage{amsfonts}

\setcounter{MaxMatrixCols}{10}
%TCIDATA{TCIstyle=LaTeX article (bright).cst}

%TCIDATA{OutputFilter=LATEX.DLL}
%TCIDATA{Version=5.00.0.2606}
%TCIDATA{<META NAME="SaveForMode" CONTENT="1">}
%TCIDATA{BibliographyScheme=Manual}
%TCIDATA{Created=Thursday, September 13, 2001 12:48:34}
%TCIDATA{LastRevised=Wednesday, February 22, 2006 13:42:45}
%TCIDATA{<META NAME="GraphicsSave" CONTENT="32">}
%TCIDATA{<META NAME="DocumentShell" CONTENT="Exams and Syllabi\SW\Assignment">}
%TCIDATA{Language=American English}

\setlength{\topmargin}{-1.0in}
\setlength{\textheight}{9.25in}
\setlength{\oddsidemargin}{0.0in}
\setlength{\evensidemargin}{0.0in}
\setlength{\textwidth}{6.5in}
\def\labelenumi{\arabic{enumi}.}
\def\theenumi{\arabic{enumi}}
\def\labelenumii{(\alph{enumii})}
\def\theenumii{\alph{enumii}}
\def\p@enumii{\theenumi.}
\def\labelenumiii{\arabic{enumiii}.}
\def\theenumiii{\arabic{enumiii}}
\def\p@enumiii{(\theenumi)(\theenumii)}
\def\labelenumiv{\arabic{enumiv}.}
\def\theenumiv{\arabic{enumiv}}
\def\p@enumiv{\p@enumiii.\theenumiii}


\parindent=0pt

\makeatother

\usepackage{babel}
\begin{document}
\begin{center}
\textbf{151-0530-00L, Spring, 2022}
\par\end{center}

\begin{center}
\textbf{Nonlinear Dynamics and Chaos II}
\par\end{center}

\begin{center}
\textbf{Homework Assignment 1}
\par\end{center}

\begin{center}
Due: Monday, March 28\\
Please email PDF file to Dr. Mattia Cenedese <mattiac@ethz.ch>
\par\end{center}

\bigskip{}

\begin{enumerate}
\item Show that the number of $k$-periodic orbits for the Bernoulli shift
map on two symbols is
\[
N(k)=\frac{1}{k}\left(2^{k}-\sum_{(i,k)}i~N(i)\right),
\]
where $(i,k)$ means that the integer $i$ divides the integer $k$.
\item Let $A$ denote the transition matrix for a sub-shift $\sigma\colon\Sigma_{A}^{N}\mapsto\Sigma_{A}^{N}$
of finite type on $N$ symbols. 

\begin{enumerate}
\item Show that the number of fixed points of $\sigma$ is equal to $\mathrm{trace(}A)$. 
\item Show that the total number of \emph{admissible} $k-$periodic points
(i..e, $k-$periodic points whose minimal period may be less than
$k$) is equal to $\mathrm{trace}(A^{k})$. 
\end{enumerate}
\item Show that any two periodic orbits of the Bernoulli shift map are connected
by infinitely many heteroclinic orbits.
\item Show that the Bernoulli shift map $\sigma$ is topologically transitive
on the symbol space $\Sigma$ with respect to the metric $d(\cdot~,\cdot)$
defined in class. Specifically, show that for any two open sets $A,B\subset\Sigma$,
there exists an integer $N$ such that $\sigma^{N}(A)\cap B\neq\emptyset$.
(\emph{Hint}: Use the existence of a dense orbit for $\sigma$ in
$\Sigma$: there exists a symbol sequence $s^{*}\in\Sigma$ with the
following property: for any $s\in\Sigma$ and for any $\delta>0$,
there exists an integer $N(s,\delta)$ such that $d(\sigma^{N(s,\delta)}(s^{*}),s)<\delta$.)
\item Show that the Bernoulli shift map $\sigma$ has sensitive dependence
on initial conditions on the symbol space $\Sigma$. Specifically,
show that there exists a nonzero distance $\Delta>0$, such that no
matter how close two symbols $s^{\ast}$ and $\bar{s}$ are in $\Sigma$,
we have
\[
d\left(\sigma^{N}(s^{\ast}),\sigma^{N}(\bar{s})\right)>\Delta
\]
for some $N.$
\end{enumerate}

\end{document}
