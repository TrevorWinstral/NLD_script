\documentclass[twoside,10pt,a4paper]{article}
\usepackage[utf8]{inputenc}
\usepackage[english]{babel}
\usepackage{amsmath}
\usepackage{amsfonts}
\usepackage{amssymb}
\usepackage{graphicx}

\usepackage[left=2cm,right=2cm,top=2cm,bottom=3cm]{geometry}
\usepackage{fancyvrb}
\usepackage{listings}
\usepackage{xparse}
\usepackage{tikz} % ajout de dessins LaTeX
\usepackage{graphicx}
\usepackage{float}  % alignement des figures
\usepackage{fancyhdr}
\usepackage{enumitem}
\usepackage{verbatim}
\usepackage{xcolor}
\usepackage{mathtools}
\usepackage{cancel}


\usepackage{caption}
\usepackage{subcaption}

\pagestyle{fancy} %fancyhdr
	\fancyhf{} %fancyhdr
	\renewcommand{\sectionmark}[1]{\markboth{#1}{}}
	\fancyhead[R]{Exercice Set 2 - Questions} %INSERT TITLE HERE FOR fancyhdr
	\fancyhead[L]{\nouppercase{\leftmark}} %fancyhdr
	\cfoot{\thepage} %fancyhdr
	\setlength{\headheight}{35pt}
	\setlength{\parindent}{0pt}
	
	\definecolor{MyBlue}{HTML}{4A90E2}
	\definecolor{MyRed}{HTML}{D0021B}

\begin{titlepage}
\title{\huge \textbf{Nonlinear Dynamics and Chaos I \\ \Large  Exercice Set 2 - Questions}}	%TITLE
\author{ }		%AUTHOR
\date{ }	%DATE

\end{titlepage}


\begin{document}

\maketitle

\section*{Question 1}
Consider the nonlinear oscillator
\begin{equation*}
	\ddot{x} + \omega_0^2x = \varepsilon Mx^2,
\end{equation*}
where $\varepsilon M x^2$ represents a small nonlinear forcing term $(0 \leq \varepsilon \ll 1, M > 0)$.

Using Lindstedt's method, find an $\mathcal{O}(\varepsilon)$ approximation for nonlinear motions as a function of their initial position, with zero initial velocity.

\section*{Question 2}
Consider the forced \textit{van der Pol equation}
\begin{equation*}
	\ddot{x} + \varepsilon(x^2 - 1)\dot{x} + x = F\cos(\omega t),
\end{equation*} 
which arises in models of self-excited oscillation, such as those of a valve generator with a cubic valve characteristic. Here $F, \omega > 0$ are parameters, and $0 \leq \varepsilon \ll 1$.

\begin{enumerate}[label=(\roman*)]
	\item For small values of $\varepsilon$, find an approximation for an \textbf{exactly} $2\pi/\omega$-periodic solution of the equation. The error of your approximation should be $\mathcal{O}(\varepsilon)$.
	\item For $\varepsilon = 0.1, \; \omega = 2$, and $F = 1$, verify your prediction numerically by solving the equation numerically. Plot your numerical solution along with your analytic prediction computed in (i).

\textit{Note}: For chaotic dynamics in the forced van der Pol equation, see Section 2.1 of \textit{Guckenheimer \& Holmes}.
\end{enumerate}









\end{document}