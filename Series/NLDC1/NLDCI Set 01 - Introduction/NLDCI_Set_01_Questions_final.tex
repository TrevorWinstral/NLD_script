\documentclass[twoside,10pt,a4paper]{article}
\usepackage[utf8]{inputenc}
\usepackage[english]{babel}
\usepackage{amsmath}
\usepackage{amsfonts}
\usepackage{amssymb}
\usepackage{graphicx}

\usepackage[left=2cm,right=2cm,top=2cm,bottom=3cm]{geometry}
\usepackage{fancyvrb}
\usepackage{listings}
\usepackage{xparse}
\usepackage{tikz} % ajout de dessins LaTeX
\usepackage{graphicx}
\usepackage{float}  % alignement des figures
\usepackage{fancyhdr}
\usepackage{enumitem}
\usepackage{verbatim}
\usepackage{xcolor}
\usepackage{mathtools}
\usepackage{cancel}


\usepackage{caption}
\usepackage{subcaption}

\pagestyle{fancy} %fancyhdr
	\fancyhf{} %fancyhdr
	\renewcommand{\sectionmark}[1]{\markboth{#1}{}}
	\fancyhead[R]{Exercice Set 1 - Questions} %INSERT TITLE HERE FOR fancyhdr
	\fancyhead[L]{\nouppercase{\leftmark}} %fancyhdr
	\cfoot{\thepage} %fancyhdr
	\setlength{\headheight}{35pt}
	\setlength{\parindent}{0pt}
	
	\definecolor{MyBlue}{HTML}{4A90E2}
	\definecolor{MyRed}{HTML}{D0021B}
	\definecolor{MyGreen}{HTML}{7ED321} % Same color use in Mathcha

\begin{titlepage}
\title{\huge \textbf{Nonlinear Dynamics and Chaos I \\ \Large  Exercice Set 1 - Questions}}	%TITLE
\author{ }		%AUTHOR
\date{ }	%DATE

\end{titlepage}


\begin{document}

\maketitle

\section*{Question 1}
Many important properties of nonlinear dynamical systems follow from Gronwall's inequality. Assume that two positive, continuous scalar functions $u(t)$ and $v(t)$ satisfy the condition
\begin{equation*}
	u(t) \leq C + \int_{t_0}^t u(\tau)v(\tau) \, \text{d}\tau
\end{equation*}
for some constant $C \geq 0$ and for all $t \geq t_0$. Then Gronwall's inequality asserts that
\begin{equation*}
u(t) \leq Ce^{\int_{t_0}^t v(\tau) \, \text{d}\tau} 
\end{equation*}
for all $t \geq t_0$. The significance of this result is that it gives a $u(t)$-independent upper bound on the growth of $u(t)$. Using Gronwall's inequality, give an upper bound on how fast the solutions of a nonlinear ODE can separate from each other in time. In particular, show that for an ODE of the form
\begin{equation*}
	\dot{x} = f(x,t), \qquad x \in \mathbb{R}^n,
\end{equation*}
and for two solutions starting from the initial conditions $x_0$ and $\hat{x}_0$ at time $t_0$, we have
\begin{equation*}
	|x(t, x_0) - x(t, \hat{x}_0)| \leq |x_0 - \hat{x}_0|e^{L(t - t_0)},
\end{equation*}
where $L$ is a Lipschitz constant for the function $f$ over a domain containing the trajectories of the system over the time interval $[t_0, t]$.

\textit{Hint}: Substitute both solutions into the ODE, integrate the resulting two equations from $t_0$ to $t$, and estimate their normed difference.

\section*{Question 2}
Consider a pendulum that strikes an inclined wall repeatedly, as shown in Fig. \ref{Q01D01} below. Using the phase portrait of the pendulum discussed in class, sketch the trajectories in the phase space of this impact dynamical system for positive and negative values of the angle $\alpha$, when

\begin{enumerate}[label=(\roman*)]
	\item there us no loss of energy at impact
	\item the coefficient of restitution is 0.5.
\end{enumerate}
Identify the asymptotic behavior of the pendulum in each case.

\begin{figure}[H]
	\centering
	\includegraphics[scale=0.9]{Graphics/Q01D01.pdf}
	\caption{Pendulum attached to the inclined wall.}
	\label{Q01D01}
\end{figure}

\textit{Hint}: During impact, the pre-impact velocity of the pendulum is assumed to jump instantaneously to its post-impact value, while the pendulum's position remains the same.

\section*{Question 3}
\end{document}
